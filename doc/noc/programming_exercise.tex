\documentclass[a4paper,fontsize=10pt,twoside,DIV15,BCOR12mm,headinclude=true,footinclude=false,pagesize,bibtotoc]{scrbook}


\usepackage[utf8]{inputenc}
\usepackage[T1]{fontenc}

\usepackage{pslatex} % -- times instead of computer modern
\usepackage[scaled=.84]{beramono} % a sane monospace font
\usepackage{microtype}

\usepackage{url}
\usepackage{booktabs}
\usepackage{graphicx}
\usepackage{textcomp}
\usepackage{xspace}
\usepackage[usenames,dvipsnames,table]{xcolor}
\usepackage{colortbl}
\usepackage{multicol}
\usepackage{rotating}
\usepackage{subfig}
\usepackage{ulem}
\usepackage{enumerate}
\usepackage{fancyvrb}
\usepackage{framed}

% Doxygen includes
\usepackage{calc}
\usepackage{../c_doc/latex/doxygen}
\usepackage{makeidx}
\usepackage{multirow}

% avoid clubs and widows
\clubpenalty=10000
\widowpenalty=10000

% tweak float placement
%% \renewcommand{\textfraction}{.15}
\renewcommand{\topfraction}{.75}
%% \renewcommand{\bottomfraction}{.7}
\renewcommand{\floatpagefraction}{.75}
%% \renewcommand{\dbltopfraction}{.66}
%% \renewcommand{\dblfloatpagefraction}{.66}
\setcounter{topnumber}{4}
%% \setcounter{bottomnumber}{4}
%% \setcounter{totalnumber}{16}
%% \setcounter{dbltopnumber}{4}

\newcommand{\code}[1]{{\texttt{#1}}}
\newcommand{\codefoot}[1]{{\textsf{#1}}}
\def\figref#1{Figure~\ref{fig:#1}}

% ulem package, otherwise emphasized text becomes underlined
\normalem


\newcommand{\todo}[1]{{\emph{TODO: #1}}}
%\renewcommand{\todo}[1]{}
\newcommand{\nopublish}[1]{}

%
% generic command to comment something
%
\newcommand{\ncomment}[3]{

\textsf{\textbf{#1}} {\color{#3}#2}}

%
% commentators
%
\newcommand{\wolf}[1]{\ncomment{Wolfgang}{#1}{OliveGreen}}
\newcommand{\martin}[1]{\ncomment{Martin}{#1}{Blue}}
\newcommand{\rasmus}[1]{\ncomment{Rasmus}{#1}{Mahogany}}

% uncomment to get rid of comments
\renewcommand{\wolf}[1]{}
\renewcommand{\martin}[1]{}
\renewcommand{\rasmus}[1]{}

%
% custom colors
%
\definecolor{lightgray}{gray}{0.8}
\definecolor{gray}{gray}{0.5}

\usepackage{listings}

% general style for listings
\lstset{basicstyle=\ttfamily,keywordstyle=\ttfamily,showstringspaces=false,language=C}

\usepackage[endianness=big]{bytefield}

% long immediate in second slot
\newcommand{\lconst}{\texttt{const}_{32}}
% short immediate in ALU instruction
\newcommand{\sconst}{\texttt{Constant}_{12}}
% constant in Rs2 field
\newcommand{\rconst}{\texttt{Constant}_{5}}

% SH: to be used in text mode .. maybe we should change this to math mode?
\newcommand{\XOR}{\textasciicircum\xspace}
\newcommand{\OR}{\textbar\xspace}
\newcommand{\AND}{\&\xspace}
\newcommand{\NOT}{\texttildelow}
\newcommand{\shl}{\textless$\!$\textless\xspace}
\newcommand{\shr}{\textgreater$\!$\textgreater$\!$\textgreater\xspace}
\newcommand{\ashr}{\textgreater$\!$\textgreater\xspace}

\newcommand{\bitsunused}{\rule{\width}{\height}}
\newcommand{\bitssubclass}{\color{lightgray}\rule{\width}{\height}}

\usepackage{mdwlist}
\renewenvironment{description}%
{
\begin{basedescript}{
\desclabelstyle{\nextlinelabel}
\renewcommand{\makelabel}[1]{%
\parbox[b]{\textwidth}{\bfseries##1}%
}%
\desclabelwidth{2em}}}
{
\end{basedescript}
}

%
% allow click-able links
%
\usepackage[open]{bookmark}
\usepackage[all]{hypcap}

%
% hyperref setup (depends on bookmark/hyperref}
%
\hypersetup{
    pdftitle = {Argo programming exercise},
    pdfsubject = {Technical Report},
    colorlinks = {true},
    citecolor = {black},
    filecolor = {black},
    linkcolor = {black},
    urlcolor = {black},
    final
}

%
% document contents
%
\begin{document}

\title{The Argo software perspective}
\subtitle{A programming exercise}

\author{Rasmus Bo S{\o}rensen}

\lowertitleback{Copyright \copyright{} 2015 Technical University of Denmark
  \medskip\\
  \begin{tabular}{lp{.8\textwidth}}
    \raisebox{-12pt}{\includegraphics[height=18pt]{fig/cc_by_sa}} &
     This work is licensed under a Creative Commons Attribution-ShareAlike
     4.0 International License.
     \url{http://creativecommons.org/licenses/by-sa/4.0/}\\
  \end{tabular}
}

\frontmatter

\maketitle

\chapter{Preface}

This exercise manual is written for the course `02211 Advanced Computer Architecture' at the Techincal University of Denmark,
but is intended as a stand alone document for anybody interrested in learning to program the Argo Network-on-Chip.

This document is subject to continous development along the the platform it describes.
In case you have suggestions for improvement or find that the text is unclear and needs to be elaborated, please write to \href{mailto:rboso@dtu.dk}{rboso@dtu.dk}
The latest version of this document is contained as LaTeX source in the Patmos repository in directory
\code{patmos/doc} and can be built with \code{make noc}.

\nopublish{
\section*{Acknowledgment}

}

\tableofcontents

%\begingroup
%\let\cleardoublepage\clearpage
%\listoffigures
%\listoftables
%\lstlistoflistings
%\endgroup

\mainmatter

\chapter{Introduction}

In this document we will present the background required to efficiently program the Argo NoC.
The exercieses should give the reader a good understanding of how the Argo NoC can be utilized in a multicore application.
The reader will get experience in how to write a multicore application that uses message passing.
In the exercises we assume that the reader is familiar with the C programming language and multithreaded programming in general.
Furthermore, we assume that the reader has already run a single core application on a Patmos processor in an FPGA, refer to the Patmos handbook~\cite{patmos-handbook} for details on the patmos processor.


Chapter~\ref{chap:arch} presents the achitecture of the Argo Network-on-Chip.
Chapter~\ref{chap:api} describes the programming interface of the multicore platform, including the thread library and the high level message passing.
Chapter~\ref{chap:exercise} contains the practical exercises ment to give the reader a practical introduction to the platform.
Appendix~\ref{apx:run} describes the practical aspects of loading the program into the platform running in an FPGA.
Appendix~\ref{apx:api} contains the Doxygen documentation of the C libraries.


\chapter{The Architecture of Argo}
\label{chap:arch}
The Argo Network-on-Chip (NoC) is a statically scheduled
time division multiplexed (TDM) NoC that supports
direct memory access (DMA) driven write transactions.

The statically allocated route and the TDM schedule is loaded
into hardware tables in the network interface at boot time.
The statically allocated route of each packet sent through the
network, is stored in the header of each packet.
The write address of each packet is also stored in the header.

The end-points of these write transactions are the dual-ported
processor local communication scratchpad memories (SPM) connected
on one port directly to the processor and on the other port to the 
network interface, containing the DMA controllers.
Figure~\ref{fig:argo-arch} shows the architecture of the Argo NoC.
\begin{figure}[h]
\centering
\includegraphics[width=\textwidth]{fig/argo-arch.pdf}
\caption{The Argo architecture from a software perspective.
A DMA write transaction moves the specified block of data from the
communication SPM of the processor on the left to the specified
location in the communication SPM of the processor on the right.}
\label{fig:argo-arch}
\end{figure}
The DMA block in the figure contains a table of DMA entries,
each entry describes a DMA controller that can send to a remote processor.
Each DMA controller is paired with a communication channel when the network in configured. 
To transfer a block of data from a local SPM to a remote SPM, there are \ref{arch:list1}~steps:
\begin{enumerate}
\item Store the block of data in the local SPM
\item Through the network interface set up the DMA controller that is paired with the correct communication channel
  \begin{itemize}
  \item Write the local address of the block of data and the remote address to which the block of data should be moved
  \item Write the size of the block of data
  \item Set the `valid' bit in the DMA antry to 1 and clear the `done' bit setting it to 0
  \end{itemize}
\label{arch:list1}\end{enumerate}
After step~\ref{arch:list1} the DMA will start to transfer data in each TDM slot that is allocated to the given channel.
When the DMA has transfered all packets through the network the `done' bit is set to 1 in the DMA entry. This `done' bit can be pooled to wait for the DMA to be done.

Conflicts of reading and writing to the same addresses in the dual
ported SPMs has to be handled by software, there is no protection in hardware.


% % % % % % % % % % % % % % % % % % % % % % % % % % % % % % % % % % % % % % % %
\chapter{Argo Programming Interface}
\label{chap:api}
In the sections of this chapter we will give an overview of each part of the Argo programming interface.
For detailed documentation refer to Appendix~\ref{apx:api}, which contains the doxygen documentation of the C libraries.

\section{Corethread Library}
\label{sec:cthread}
When an application is downloaded to the platform, the \code{main()} function is executed on the master core, the master core has the core id 0.
From the \code{main()} function the programmer can execute functions on the slave cores using the functions in the \code{libcorethread} library.
The functions of the \code{libcorethread} library are:
\begin{description}
\item[\code{int corethread\_create( corethread\_t *thread, void(*start\_routine)(void *), void *arg ) )}]

The create function will start the execution of the \code{start\_routine} function on the core specified by \code{thread}, an argument can be given to the started function via the \code{arg} pointer. The start function should only be called by the master core during the initialization phase of the application.

\item[\code{void corethread\_exit( void * retval )}]

The exit function can be called in the \code{start\_routine} functions if they need to return a value to the master core.
The exit function should be called as the last thing before the return statement.

\item[\code{int corethread\_join( corethread\_t thread, void ** retval )}]

The join function will join the program flow of the master core with the program flow of the core specified by \code{thread}, the join function should only be called from the master core. The join function will point the \code{retval} pointer to the return value allocated by the thread on the slave core. Be aware, the return value should not be allocated on the stack of the slave core!

\end{description}


\section{NoC Driver}
The Argo NoC has a very minimalistic interface.
Initialization of the NoC is done before the \code{main()} function starts executing.
To hide the direct interfacing to the hardware, we use the \code{libnoc} that is a driver for the NoC.
If the application requires direct control over data movement through the NoC the three functions presented in the following can be used, otherwise the application should use the message passing library presented in Section~\ref{sec:libmp}.

\begin{description}
\item[\code{int noc\_done( unsigned rcv\_id )}]

The done function is used to tell whether a local DMA transfer has finished.

\item[\code{int noc\_nbsend( unsigned rcv\_id, volatile void \_SPM *dst, volatile void \_SPM *src, size\_t size )}]

The nbsend function is a non-blocking function for sending a block of data at the address \code{src} of size \code{size} to the core with the core id \code{rcv\_id} and the remote address \code{dst}. The nbsend function will fail if the DMA controller is still sending the previous block of data.

\item[\code{void noc\_send( unsigned rcv\_id, volatile void \_SPM *dst, volatile void \_SPM *src, size\_t size )}]

The send function is calling the nbsend in a while loop, until it returns success.

\end{description}

\section{Message Passing Library}
\label{sec:libmp}

The Argo interface (libnoc) does not support receive notification,
flow control or double buffering.
This functionality is implemented in the \code{libmp} library.
\code{libmp} implements message passing with statically allocated buffers
that does not require data to be copied unnecessarily.

\begin{description}
\item[\code{int mp\_chan\_init( mpd\_t* mpd\_ptr, coreid\_t sender, coreid\_t receiver, unsigned buf\_size, unsigned num\_buf )}]

The chan\_init function allocates the static buffer structures of a communication channel
and initializes the message passing descriptor \code{mpd\_ptr}.
The communication channel is set up between the sending
core \code{sender} and the receviing core \code{receiver}.
The communication channel will transfer messages of size
\code{buf\_size} and buffer a number of \code{num\_buf} messages in the receiver SPM.

\item[\code{int mp\_nbsend( mpd\_t* mpd\_ptr )}]

The nbsend function checks if there is a free buffer in the receiver
and if the DMA controller for the given communication channel is free.
If both are free it will set up the DMA to transfer the new block of data.
The nbsend function assumes that the user/application already
wrote the data to be sent into the \code{write\_buf} buffer.

\item[\code{void mp\_send( mpd\_t* mpd\_ptr )}]

The send function calls the nb\_send function in a loop, until it returns success.

\item[\code{int mp\_nbrecv( mpd\_t* mpd\_ptr )}]

The nbrecv function checks if the next buffer, in the buffer queue,
has received a complete message.
If a message is received it will move the \code{read\_buf} pointer
to the beginning of the message,
such that the user/application can read the received data.

\item[\code{void mp\_recv( mpd\_t* mpd\_ptr )}]

The recv function calls the nb\_recv function in a loop, until it returns success.

\item[\code{int mp\_nback( mpd\_t* mpd\_ptr )}]

The nback function increment the number of messages that
has been acknowledged and sends the updated value to the 
sender core, if the send does not succeed the number
of acknowlegded messages is decremented.

\item[\code{void mp\_ack( mpd\_t* mpd\_ptr )}]

The ack function calls the nb\_ack function in a loop, until it returns success.

\item[\code{void \_SPM * mp\_alloc( coreid\_t id, unsigned size )}]

The alloc function will allocate a block of memory of size
\code{size} in the SPM local to the core with the id \code{id}.
The alloc function can only be called from the master core executing \code{main()}
and once the memory block is allocated it cannot be freed.
In the current version of the software the alloc function will not
give an out of memory error, so the programmer should be aware
that he/she does not allocate more local memory than is present.

\end{description}

% % % % % % % % % % % % % % % % % % % % % % % % % % % % % % % % % % % % 

\chapter{Exercises}
\label{chap:exercise}

The following exercises are made to run on the default
9 core platform for the Altera DE2-115 board.
Please refere to the Appendix~\ref{apx:buildhw} for instructions on
how to build an up-to-date hardware platform.

\section{Circulating tokens}
By creating an application that mimics streaming behavior between a number of processors, this exercise will illustrate to the reader how the basics of message passing works on Argo.
In this exercise we will make an application that circulates a number of tokens in a ring of 8 slave processors.
The number of tokens should be configurable, but always less than the number for processors.
Each of the processors in the ring shall repeatedly execute the following \ref{list:ex1:num1}~steps:

\begin{framed}
\begin{enumerate}
\item Receive a token from the previous processor

\item Turn on the processor LED to indicate that the token is beeing processed

\item Wait for a random amount of time in the interval [100 ms; 1 s]
\label{list:step_rand}
\item Send the token to the next processor, when the send is complete Turn off the processor LED to indicate the the token has been processed

\label{list:ex1:num1}\end{enumerate}
\end{framed}

\noindent Looking at the LEDs when the application runs,
the reader should see tokens move from one led to the other.
This behavior should be easy to observe with only few tokens.
This exercise is split into \ref{list:ex1:num2}~tasks:
\begin{framed}
\begin{enumerate}
\item Create a function that blinks an LED and create a thread on each slave core that executes the blink function
\item Extend the blink function to turn the LED on and off at random times
\item Extend the blink function to receive a message from the previous core in the ring and send a message to the next core in the ring
\item Change the blink function such that it sends the random seed value along with the token
\label{list:ex1:num2}\end{enumerate}
\end{framed}

\noindent In each task you should verify the your program is working as expected by compiling and downloading the program to the platform.

\subsection{Task 1}
In this task you should create a function that blinks the LED and execute the function on the slaves processors.
The frequency of blinking the LED should be in the order of 1~-~10~Hz so that it is visible to the eye.
We suggest to set the frequency of the blinking through a parameter of the blinking function.
Figure~\ref{fig:ctrl_led} shows an example of how to blink an LED.
To turn the LED on and off, write a 1 and 0, respectively to the hardware address of the LED. 

\begin{figure}
\begin{Verbatim}[xleftmargin=1cm,xrightmargin=1cm,frame=single,framesep=3mm]
// The hardware address of the LED
#define LED ( *( ( volatile _IODEV unsigned * ) 0xf0000900 ) )
for (;;) {
  for ( int i=2000; i!=0; --i ) {
    for ( int j=2000; j!=0; --j ) {
      // Turn off LED
      LED = 1;
    }
  }

  for ( int i=2000; i!=0; --i ) {
    for ( int j=2000; j!=0; --j ) {
      // Turn off LED
      LED = 0;
    }
  }
}
\end{Verbatim}
\caption{\label{fig:ctrl_led}An example of how to blink an LED}
\end{figure}

To execute the \code{blink()} function on the slave core there is an example of
how to call the \code{corethread\_create()} function in Figure~\ref{fig:corethread}.
Section~\ref{sec:cthread} explains in further detail, how corethreads are started
on slave processors and how a parameter can be passed to the function.

\begin{figure}
\begin{Verbatim}[xleftmargin=1cm,xrightmargin=1cm,frame=single,framesep=3mm]
void loop(void* arg) {
  int num_tokens = *((int*)arg);
  /*
    Write code in the slave loop
  */
}

int main() {
  corethread_t worker_id = 1; // The core ID
  int parameter = 42;
  corethread_create( &worker_id, &loop, (void*) &parameter );  

  int* res;
  corethread_join( worker_id, &res ); // No return value is returned

  return *res;  
}
\end{Verbatim}
\caption{\label{fig:corethread}An example of how to create a corethread.}
\end{figure}

\paragraph*{Expected output}
The 8 LEDs on the board should all blink with the specified frequency.

\subsection{Task 2}
In task 2 you shall extend the blink function from task 1 to turn the LED on and off at random times.
We suggest to use the \code{rand\_r()} function to generate a random number,
\code{rand\_r()} takes a pointer to a seed value in order to generate a random number.
Do not use the \code{rand()} function as it is not thread-safe.
The seed value in each core should be different, otherwise all cores have the
same sequence of psedo-random numbers.

Use the lower bits of the random number to generate a number on the desired range.
The \code{get\_cpu\_usecs()} function returns the value of the microsecond counter as an \code{unsigned long long}. 

\paragraph*{Expected output}
The 8 LEDs should now independently blink with random varying frequencies.

\subsection{Task 3}
In this task you will start sending messages.
Before messages can be sent or received you need to initialize the message passing channels with the \code{mp\_chan\_init()} function.
The use of the message passing function is described in Section~\ref{sec:libmp}.
The initialization of the message passing channels shall be done in the \code{main()} function before the corethreads are started.
If the channels are initialized after the corehtreads are created the behavior of
\code{mp\_send()} and \code{mp\_recv()} is undefined.


\paragraph*{Expected output}
It should now be observable that the tokens move between cores.

\subsection{Task 4}
For the sake of the example, you should now pair a seed value to each tokens.
To send the seed value along with the message, you need to write the seed value into
the \code{write\_buf} before sending the message, and read out the seed value from the \code{read\_buf}
after receiving a message.
Figure~\ref{fig:msg_data} shows an example of how message data can be read and written.

\begin{figure}
\begin{Verbatim}[xleftmargin=1cm,xrightmargin=1cm,frame=single,framesep=3mm]
// Reading an unsigned integet value from the channel read buffer
seed = *( ( volatile unsigned int _SPM * ) ( &chan.read_buf ) );

// Writing an unsigned integer value to the channel write buffer
*( volatile unsigned int _SPM * ) ( &chan.write_buf ) = seed; 
\end{Verbatim}
\caption{\label{fig:msg_data}An example of how to read and write message data}
\end{figure}

\paragraph*{Expected output}
The output should not be different form task 3.

\subsection{Extensions}
If you have more time left or just can not get enough of programming message passing applications, you can extend your application  in several ways:
\begin{itemize}
\item Move the calculation of random numbers to core 0.
Core 0 shall act like a server replying with a new random number when it receives a message from any of the slave cores.
\item Create a mechanisme that terminates the executution of the blink function on the slaves, when the master is signaled to stop thought the terminal. 
\end{itemize}

\nopublish{

\section{Circulating tokens - WCET}

This exercise will show how to calculate the maximum worst case throughput of tokens.
The exercise comprise the following tasks:
\begin{itemize}
\item Compile with platin
\item 
\label{list:ex2:num1}\end{itemize}

\section{Next exercise}

Ideas to other exercises:
\begin{itemize}
\item WCET analysis of the circulating tokens
  \begin{itemize}
  \item Worst case throughput of tokens
  \item Worst case latency of a token through two processors
  \end{itemize}
\item Health monitoring 
\item I/O server
\end{itemize}

}
%%%%%%%%%%%%%%%%%%%%%%%%%%%%%%%%%%%%%%%%%%%%%%%%%%%%%%%%%%%%
% The start of the appendix
\appendix

%%%%%%%%%%%%%%%%%%%%%%%%%%%%%%%%%%%%%%%%%%%%%%%%%%%%%%%%%%%%%%%%%%%%%%%%%%%%%%
\chapter{Build And Execute Instructions}
\label{apx:build}

In this chapter we present the details on how to 
build and configure the hardware platform and compile and execute 
a multicore program on the platform.

\section{Build and configure the hardware platform}
\label{apx:buildhw}

The Aegean framework generates a hardware description from an
xml description.
The default xml description for the Altera DE2-115 board with 9 cores
has an external shared memory and an Argo network-on-chip.
To build the newest version of the platform run the following commands:
\begin{Verbatim}
cd ~/t-crest/aegean
git pull
make AEGEAN_PLATFORM=default-altde2-115-9core platform synth
\end{Verbatim}
The \code{git pull} will update the Aegean framework with the newest
configuration files.
The make command will generate a platform as described in the
\code{config/default-altde2-115-9core.xml} file.
When the platform description is generated the it will be synthesised.
When the synthesis is finished the multicore platform can be
configured into the FPGA using the folowing commands:
\begin{Verbatim}
cd ~/t-crest
make -C aegean AEGEAN_PLATFORM=default-altde2-115-9core config
\end{Verbatim}

\section{Compile and execute a multicore program}
\label{apx:buildsw}

There is no difference between compiling a single core program and a multicore program.
Furthermore, a single core program can execute in a multicore
platform without any modifications.
To compile a multicore program, place it in the \code{patmos/c/} directory
and run the following commands:

\begin{Verbatim}
cd ~/t-crest
make -C patmos APP=${APP_NAME} comp
\end{Verbatim}

The \code{comp} target will compile the C program in the file \code{patmos/c/\$\{APP\_NAME\}.c}
and output an .elf file \code{patmos/tmp/\$\{APP\_NAME\}.elf}.
When compiling a program that includes either \code{``libmp/mp.h''} or \code{``libnoc/noc.h''},
the \code{nocinit.c}, generated by the Aegean framework, is included needed,
as this contains the configuration data for the Argo NoC.
To download the program to the configured FPGA, run the following commands:

\begin{Verbatim}
cd ~/t-crest
make -C patmos APP=${APP_NAME} download
\end{Verbatim}

The \code{download} target of the Makefile depends on the comp target,
therefore it is not necessary to execute the \code{comp} target before every download.
Also, it is not strictly necessary to configure the FPGA with the hardware
platform between each download of a program, but we advice you to do so.
This will ensure that the hardware platform is probably iniitaliized
before your download a program.


% % % % % % % % % % % % % % % % % % % % % % % % % % % % % % % % % % % % % % % %
\chapter{Library Documentation}
\label{apx:api}
\section{Module Documentation}
\input{../c_doc/latex/group__libcorethread}
\input{../c_doc/latex/group__coreset}
\input{../c_doc/latex/group__libnoc}
\input{../c_doc/latex/group__libmp}
\section{Class Documentation}
\input{../c_doc/latex/structcommunicator__t}
\input{../c_doc/latex/structcoreset__t}
\input{../c_doc/latex/structmpd__t}
\section{File Documentation}
\input{../c_doc/latex/coreset_8h}
\input{../c_doc/latex/corethread_8h}
\input{../c_doc/latex/mp_8h}
\input{../c_doc/latex/noc_8h}

\bibliographystyle{abbrv}
\bibliography{argobib.bib}

\end{document}
